\startproject iras
\mainlanguage[hu]
\setupalign[
  justified,
  nothanging,
  nohz,
  hyphenated,
  morehyphenated,
  tolerant,
]
\setupinterlinespace[height=0.75,depth=0.25]
\setuplayout[
  grid=no,
  location=middle,
]
\setupformulas[align=middle]
%\setupmathalignment[grid=no]

\def\PagenumberingCommand#1{\doifnot\pagenumber1{#1}}
\setuppagenumbering[
  location={footer,middle},
  command=\PagenumberingCommand,
]
\setuppapersize[A4]
\setuplayout[
    backspace=30mm,
    width=150mm,
    topspace=30mm,
    header=0mm,
    footer=15mm,
    footerdistance=0mm,
    bottom=0mm,
    bottomdistance=0mm,
    height=247mm
]

% Betűkészlet
%\setupbodyfont[libertinus,12pt]
\setupbodyfont[12pt]

% Vékony spácium bizonyos karakterek előtt (:;?!)
\definecharacterspacing [magyarpunctuation]
\setupcharacterspacing [magyarpunctuation] ["0021] [left=.1,alternative=1] % ! % strip preceding space(char)
\setupcharacterspacing [magyarpunctuation] ["003A] [left=.1,alternative=1] % : % strip preceding space(char)
\setupcharacterspacing [magyarpunctuation] ["003B] [left=.1,alternative=1] % ; % strip preceding space(char)
\setupcharacterspacing [magyarpunctuation] ["003F] [left=.1,alternative=1] % ? % strip preceding space(char)

% A magyar nyelv beállításai
\startsetups[magyar]
  % Vékony spácium bizonyos karakterek előtt (:;?!)
  \setcharacterspacing[magyarpunctuation]
  \setupindenting[%
    yes,% A bekezdéseket behúzással kezdjük.
    %next,% Az első bekezdés nincs behúzva.
    medium% Közepes méretű (átmeneti megoldás: igazából a mérete 24 cicerós sorig 1 kvirt, nagyobbbnál 2 kvirt kellene legyen -> TENNIALÓ)
  ]
\stopsetups

\setuplanguage[hu][%
  setups=magyar,% Érvényesíti a fent megadott beállításokat.
  spacing=packed% Frenchspacing (Gyurgyák 319. o.: egyenletes szóközök).
                % http://wiki.contextgarden.net/French_spacing).
]

% TENNIVALÓ: csak magyar nyelvre
% Idézetek (Gyurgyák, 86--87. o.).
\definedelimitedtext[quote][location=text]
\setupdelimitedtext[quote:1][
  left={\lowerleftdoubleninequote},
  right={\upperrightdoubleninequote},
  spaceafter=0
]
\setupdelimitedtext[quote:2][
  left={\rightguillemot\nobreak\hskip-.07em},
  right={\kern-0.03em\leftguillemot},
  spaceafter=0
]
\setupdelimitedtext[quote:3][
  left={\upperleftsingleninequote},
  right={\upperrightsingleninequote},
  spaceafter=0
]

\definebodyfontenvironment[default][em=italic]

\defineframedtext[kerdes][align=center,offset=0.5ex,style=italic,width=\dimexpr0.8\dimexpr\makeupwidth]

\defineframedtext[kivonat][offset=0.5ex, frame=off,style=italic,width=\dimexpr0.8\dimexpr\makeupwidth]

\definehead[feladat][subsubsubject]
\setuphead[feladat][
    number=no,
    textdistance=0pt,
    alternative=text,
    style=bf,
    commandafter={\hbox{.\quad}},
]

\definebar[mathubar][underbar]  % enélkül esetleg elcsúszik az aláhúzás

%\showgrid[all]

\definehead[cim][chapter]
\setuphead[cim][number=no,align=middle,after={},]

\startproduct iras
\startbodymatter

\cim{Összeadás és kivonás}

\startkivonat
Az alábbi szöveg Obádovics J{.} Gyula {\em Matematika} című könyvéből származik. Néhány kisebb módosítás, kiegészítés viszont történt.
\stopkivonat

Két munkás közül az egyik egy óra alatt 34 szegecset illesztett be, a másik 27-et. Hány szegecset illesztettek be együtt egy óra alatt?

Annyi szegecset illesztettek be, mint amennyi 34 és 27 együttvéve, amelynek a matematikai írásmódja: $34+27=61$ (olv.: harmincnégy {\em plusz} huszonhét egyenlő hatvaneggyel). Ez a művelet az összeadás, műveleti jele: $+$. A 34-et és a 27-et összeadandóknak vagy {\em tagoknak}, az eredményt, a 61-et pedig {\em összegnek} nevezzük.

\startformula \startmathalignment[n=1]
\NC \text{összeadandó} + \text{összeadandó} = \text{összeg} \NR[+]
\NC \text{tag} + \text{tag} = \text{összeg} \NR[+]
\stopmathalignment \stopformula

Az összeadás első fontos tulajdonsága, hogy a {\em tagok sorrendje felcserélhető}, az összegük nem változik, azaz az összeg független az összeadás sorrendjétől. Ez az összeadás felcserélési (kommutatív) törvénye.

\startformula 3+5=5+3=8 \stopformula

Az összeadás második fontos tulajdonsága, hogy a tagokat tetszés szerint csoportosíthatjuk, az összeg nem változik. Például három tag esetén bármelyik kettő összegéhez hozzáadhatjuk a harmadik tagot, az összeg nem változik. Ez az összeadás csoportosítási (asszociatív) törvénye.

\startformula (3+5)+7=3+(5+7)=15 \stopformula

Az összeadást többjegyű összeadandók esetén a következő módon végezzük: a számokat úgy írjuk egymás alá, hogy az azonos helyi értékű számjegyek egymás alá kerüljenek.
Az utolsó összeadandót aláhúzzuk. Például:

\startformula
%{\framed[width=\textwidth,frame=off,align=middle,after={\blank}]{
{\framed[frame=off,align={flushright,nothyphenated,verytolerant},before={},after={}]{
$2435$\\
$247$\\
\underbar{$+1413$}\\
$4095$\\
}}
%}}
\stopformula

Az összeadást a legkisebb helyi értékű jegyenél kezdjük: $3+7+5=15$.
Az eredmény $5$ egyesét leírjuk az egyesek alá, $1$ tizesét pedig a tízesekhez adjuk hozzá: $1+1+4+3=9$.
A százasok összege: $4+2+4=10$, tehát a százasok oszlopa alá $0$-t írunk, és az $1$ ezrest az ezresekhez adjuk hozzá: $1+1+2=4$.
A $4$-et az ezresek alá írjuk. 
A három adott többjegyű szám összege tehát 4095.
Az összeadást felülről lefelé is végezzük el, mert számításunk helyességét így ellenőrizhetjük.

15\,m vörösréz huzalból elhasználtunk 9\,m-t. Hány méter maradt?

Annyi maradt, amennyivel több a 15\,m, mint a 9\,m, ennek matematikai írásmódja: $15-9=6$ (olv.: tizenöt {\em mínusz} kilenc egyenlő hattal).
Ez a művelet a kivonás, műveleti jele: $-$.

Azt a számot kerestük, amelyhet 9-hez adva összeül 15-öt kapunk.
A kivonás az összeadás fordított művelete. Kivonásnál ismert az egyik összeadandó és az összeg és keressük a másik összeadandót. Az összeget {\em kisebbítendőnek}, az ismert összeadandót {\em kivonandónak}, az ismeretlen összeadandót pedig {\em különbségnek} vagy {\em maradéknak} nevezzük.

\startformula \startmathalignment[n=1]
\NC \text{kisebbítendő} - \text{kivonandó} = \text{különbség} \NR[+]
\NC \text{kisebbítendő} - \text{kivonandó} = \text{maradék} \NR[+]
\stopmathalignment \stopformula

A kivonás helyes elvégzésének próbája az összeadás:

\startformula \text{kivonandó} + \text{különbség} = \text{kisebbítendő} \stopformula

Fontos tulajdonsága a kivonásnak, hogy a különbség értéke nem változik, ha a kisebbítendőhöz és a kivonandóhoz {\em ugyanazt a számot hozzáadjuk, vagy mindkettőből ugyanazt kivonjuk}.

\startformula \startmathalignment[n=1]
\NC 13-11=2 \NR[+]
\NC (13+3)-(11+3)=16-14=2 \NR[+]
\NC (13-3)-(11-3)=10-8=2 \NR[+]
\stopmathalignment \stopformula

A kisebbítendő növelésével ill{.} csökkentésével a különbség ugyanannyival nő ill{.} csökken. A kivonandó növelésével a különbség ugyanannyival csökken, csökkentésével ugyanannyival nő.

Több számjegyű számok kivonását úgy végezzük el, hogy a számokat helyi értéküknek megfelelően egymás alá írjuk és a kivonást a legkisebb helyi értékű jegynél kezdjük:

\startformula
%{\framed[width=\textwidth,frame=off,align=middle,after={\blank}]{
{\framed[frame=off,align={flushright,nothyphenated,verytolerant},before={},after={}]{
$5832$\\
\underbar{$-3521$}\\
$2311$\\
}}
%}}
\stopformula

1 meg 1 az 2 (leírjuk az 1-et), 2 meg 1 az 3 (leírjuk az 1-et), 5 meg 3 az nyolc (leírjuk a 3-at), 3 meg 2 az öt (leírjuk az 2-t). A különbség: 2311.

Ellenőrzés: $2311+3521=5832$.

Amikor a kisebbítendő valamelyik számjegye kisebb, mint a kivonandó ugyanolyan helyi értékű számjegye, akkor felhasználjuk a már említett törvényt, amely szerint a különbség nem változik, ha a kisebbítendőt és a kivonandót ugyanazzal a számmal növeljük.

\startformula
%{\framed[width=\textwidth,frame=off,align=middle,after={\blank}]{
{\framed[frame=off,align={flushright,nothyphenated,verytolerant},before={},after={}]{
$3762$\\
\underbar{$-1835$}\\
$1927$\\
}}
%}}
\stopformula

Mivel 2 egyesből 5 egyest nem vonhatunk ki, a kisebbítendőhöz 10 egyest hozzáadunk, így 12 egyesünk lesz, és hogy a különbség ne változzék, a kivonandóhoz ugyancsak 10 egyest, de 1 tízes alakjában hozzáadunk a kivonandó tízeseihez is. Hasonlóan járunk el a százasoknál is, ahol 10 százassal (1 ezressel) növeljük a kisebbítendőt és 1 ezressel a kivonandót. A kivonást tehát így végezzük: 5 meg 7 az 12 (leírjuk a 7-et), 1 meg 3 az 4 meg 2 az 6 (leírjuk a 2-t), 8 meg 9 az 17 (leírjuk a 9-et), 1 meg 1 az 2, 2 meg 1 az 3 (leírjuk az egyet). A különbség vagy maradék: 1927. 

Próba: ha a maradékot a kivonandóhoz hozzáadva a kisebbítendőt kapjuk, helyes az eredményünk.

\startformula
%{\framed[width=\textwidth,frame=off,align=middle,after={\blank}]{
{\framed[frame=off,align={flushright,nothyphenated,verytolerant},before={},after={}]{
$1835$\\
\underbar{$+1927$}\\
$3762$\\
}}
%}}
\stopformula

Az $\mathbb{N}$ (természetes számok) számhalmazban a kivonást csak akkor végezhetjük el, ha a $\text{kisebbítendő}\geq\text{kivonandó}$ feltétel teljesül.

\stopsubject

\stopbodymatter
\stopproduct
\stopproject
