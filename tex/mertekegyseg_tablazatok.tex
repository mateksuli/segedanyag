\startproject iras
\mainlanguage[hu]
\setupalign[
  justified,
  nothanging,
  nohz,
  hyphenated,
  morehyphenated,
  tolerant,
]
\setupinterlinespace[height=0.75,depth=0.25]
\setuplayout[
  grid=no,
  location=middle,
]
\setupformulas[align=middle]
%\setupmathalignment[grid=no]

\def\PagenumberingCommand#1{\doifnot\pagenumber1{#1}}
\setuppagenumbering[
  location={footer,middle},
  command=\PagenumberingCommand,
]
\setuppapersize[A4]
\setuplayout[
    backspace=30mm,
    width=150mm,
    topspace=15mm,
    header=0mm,
    footer=5mm,
    footerdistance=0mm,
    bottom=0mm,
    bottomdistance=0mm,
    height=267mm
]

% Betűkészlet
%\setupbodyfont[libertinus,12pt]
%\setupbodyfont[12pt]

% Vékony spácium bizonyos karakterek előtt (:;?!)
\definecharacterspacing [magyarpunctuation]
\setupcharacterspacing [magyarpunctuation] ["0021] [left=.1,alternative=1] % ! % strip preceding space(char)
\setupcharacterspacing [magyarpunctuation] ["003A] [left=.1,alternative=1] % : % strip preceding space(char)
\setupcharacterspacing [magyarpunctuation] ["003B] [left=.1,alternative=1] % ; % strip preceding space(char)
\setupcharacterspacing [magyarpunctuation] ["003F] [left=.1,alternative=1] % ? % strip preceding space(char)

% A magyar nyelv beállításai
\startsetups[magyar]
  % Vékony spácium bizonyos karakterek előtt (:;?!)
  \setcharacterspacing[magyarpunctuation]
  \setupindenting[%
    yes,% A bekezdéseket behúzással kezdjük.
    %next,% Az első bekezdés nincs behúzva.
    medium% Közepes méretű (átmeneti megoldás: igazából a mérete 24 cicerós sorig 1 kvirt, nagyobbbnál 2 kvirt kellene legyen -> TENNIALÓ)
  ]
\stopsetups

\setuplanguage[hu][%
  setups=magyar,% Érvényesíti a fent megadott beállításokat.
  spacing=packed% Frenchspacing (Gyurgyák 319. o.: egyenletes szóközök).
                % http://wiki.contextgarden.net/French_spacing).
]

% TENNIVALÓ: csak magyar nyelvre
% Idézetek (Gyurgyák, 86--87. o.).
\definedelimitedtext[quote][location=text]
\setupdelimitedtext[quote:1][
  left={\lowerleftdoubleninequote},
  right={\upperrightdoubleninequote},
  spaceafter=0
]
\setupdelimitedtext[quote:2][
  left={\rightguillemot\nobreak\hskip-.07em},
  right={\kern-0.03em\leftguillemot},
  spaceafter=0
]
\setupdelimitedtext[quote:3][
  left={\upperleftsingleninequote},
  right={\upperrightsingleninequote},
  spaceafter=0
]

\definebodyfontenvironment[default][em=italic]

\defineframedtext[kerdes][align=center,offset=0.5ex,style=italic,width=\dimexpr0.8\dimexpr\makeupwidth]

\defineframedtext[kivonat][frame=off,bodyfont=small,width=\makeupwidth]

\definehead[feladat][subsubsubject]
\setuphead[feladat][
    number=no,
    textdistance=0pt,
    alternative=text,
    style=bf,
    commandafter={\hbox{.\quad}},
]

\definebar[mathubar][underbar]  % enélkül esetleg elcsúszik az aláhúzás

%\showgrid[all]

\definehead[cim][chapter]
\setuphead[cim][number=no,align=middle,after={},]

\definehead[alcim][section]
\setuphead[alcim][number=no,align=middle,after={},]

\startproduct iras
\startbodymatter

%\cim{Mértékegység táblázatok}
%\startlinealignment[middle]
%v1.0
%\stoplinealignment
 
% \blank

% Az ember az egyszerű megszámlálhatóságon, a darabszámon túl még rengeteg mindent képes mérni és ezáltal összehasonlítani is.
% Mérünk távolságot, területet, tömeget, térfogatot, hőmérsékletet, energiát, fényt, és még elképesztően sok fizikai, kémiai, csillagászati, és más jelenséget.
% 
% Máig sem sikerült az emberiségnek egyezségre jutnia és közös mérési szabványt kialakítania, habár már talán közel az év, amikor ez bekövetkezik. Az {\em SI}, a Nemzetközi Mértékegységrendszer a tudományos életben már teljesen elfogadott, sőt a hétköznapi ember számára is ezt a rendszert alkalmazza az angolszász országokon kívül szinte mindenütt.
% 
% A {\em mennyiség} egy mérés eredményének az egyértelmű megnevezése.
% Például ha azt mondjuk, hogy egy adott buszban 80 ülőhely van, akkor ezzel mindenki számára érthető módon közöltük a busz ülőhelyek szerinti méretét.
% A mennyiség {\em mértékből} és a dolog megnevezéséből áll, aminek a mennyiségét közöljük.
% Az előbbi példában a mérték a 80 és a dolog pedig az ülőhely volt.
% 
% Ha a mérték darabszámot jelöl, akkor egyben {\em mérőszám} is.
% Ha viszont a mérték nem darabszámot jelöl, akkor két részből áll maga is: {\em mérőszámból} és {\em mértékegységből}.
% Például ha azt mondjuk, hogy 10\,m kötél, akkor a kötél mértéke a 10\,m, ebben pedig a 10 a mérőszám és az {\em m} (méter) a mértékegység.
% 
% Az {\em SI} mindössze hét alapegységet használ, és ezek adják vagy ezekből származtat minden mértékegységet.
% Ebből az ötödik osztályban a következő négyet illik ismerni:

% \blank

\defineunit[textunit][alternative=text]

\alcim{Alapegységek}

\midaligned{
\bTABLE
\setupTABLE[frame=off,loffset=1em,roffset=1em]
\setupTABLE[c][each][align={middle,lohi}]
\setupTABLE[r][1,5][bottomframe=on]
\bTR
  \bTD \bf mérték \eTD
  \bTD \bf alapegység \eTD
  \bTD \bf jel \eTD
\eTR
\HL
\bTR
  \bTD hosszúság \eTD
  \bTD méter \eTD
  \bTD m \eTD
\eTR
\bTR
  \bTD hőmérséklet \eTD
  \bTD Kelvin \eTD
  \bTD K \eTD
\eTR
\bTR
  \bTD idő \eTD
  \bTD másodperc (secundum) \eTD
  \bTD s \eTD
\eTR
\bTR
  \bTD tömeg \eTD
  \bTD kilogramm \eTD
  \bTD kg \eTD
\eTR
\bTR
  \bTD terület \eTD
  \bTD négyzetméter \eTD
  \bTD m\high{2} \eTD
\eTR
\bTR
  \bTD térfogat \eTD
  \bTD köbméter \eTD
  \bTD m\high{3} \eTD
\eTR
\eTABLE
}

%\blank[3cm]

\alcim{Köznapi mértékegységek}

\midaligned{
\bTABLE
\setupTABLE[frame=off,loffset=0.5em,roffset=0.5em]
\setupTABLE[c][each][align={middle,lohi}]
\setupTABLE[r][1][bottomframe=on]
\bTR
  \bTD \bf mérték \eTD
  \bTD \bf egység \eTD
  \bTD \bf jel \eTD
  \bTD \bf átváltás módja \eTD
\eTR
\bTR
  \bTD hőmérséklet \eTD
  \bTD Celsius \eTD
  \bTD \high{\circ}C \eTD
  \bTD $[\text{\high{\circ}C}] = [\text{K}] - 273{,}15$ \eTD
\eTR
\bTR
  \bTD hőmérséklet \eTD
  \bTD Fahrenheit \eTD
  \bTD \high{\circ}F \eTD
  \bTD $[\text{\high{\circ}F}] = \displaystyle{\fraction{9}{5}} \cdot [\text{K}] - 459{,}67$ \eTD
\eTR
\bTR
  \bTD idő \eTD
  \bTD perc \eTD
  \bTD min \eTD
  \bTD $[\text{min}] = 60 \cdot [\text{s}]$ \eTD
\eTR
\bTR
  \bTD idő \eTD
  \bTD óra \eTD
  \bTD h \eTD
  \bTD $[\text{h}] = 60 \cdot [\text{min}]$ \eTD
\eTR
\bTR
  \bTD idő \eTD
  \bTD nap \eTD
  \bTD d \eTD
  \bTD $[\text{d}] = 24 \cdot [\text{h}]$ \eTD
\eTR
\bTR
  \bTD tömeg \eTD
  \bTD tonna \eTD
  \bTD t \eTD
  \bTD $[\text{t}] = 1000 \cdot [\text{kg}]$ \eTD
\eTR
\bTR
  \bTD terület \eTD
  \bTD hektár \eTD
  \bTD ha \eTD
  \bTD $[\text{ha}] = 10000 \cdot [\text{m\high{2}}]$ \eTD
\eTR
\bTR
  \bTD térfogat \eTD
  \bTD liter \eTD
  \bTD l \eTD
  \bTD $[\text{l}] = \displaystyle{\fraction{1}{1000}} \cdot [\text{m\high{3}}] = 1 \cdot [\text{dm\high{3}}]$ \eTD
\eTR
\eTABLE
}

%\blank[3cm]
\alcim{Előtagok}

\midaligned{
\bTABLE
\setupTABLE[frame=off,loffset=1em,roffset=1em]
\setupTABLE[c][each][align={middle,lohi}]
\setupTABLE[r][7,8,9,10,11,12][height=7ex]
\setupTABLE[r][1][bottomframe=on]
\bTR
  \bTD \bf előtag \eTD
  \bTD \bf jel \eTD
  \bTD[nc=2] \bf szorzó \eTD
\eTR
\HL
\bTR
  \bTD giga \eTD
  \bTD G \eTD
  \bTD 10\high{9} \eTD
  \bTD 1000000000 \eTD
\eTR
\bTR
  \bTD mega \eTD
  \bTD M \eTD
  \bTD 10\high{6} \eTD
  \bTD 1000000 \eTD
\eTR
\bTR
  \bTD kilo \eTD
  \bTD k \eTD
  \bTD 10\high{3} \eTD
  \bTD 1000 \eTD
\eTR
\bTR
  \bTD hekto \eTD
  \bTD h \eTD
  \bTD 10\high{2} \eTD
  \bTD 100 \eTD
\eTR
\bTR
  \bTD deka \eTD
  \bTD dk \eTD
  \bTD 10\high{1} \eTD
  \bTD 10 \eTD
\eTR
\bTR
  \bTD \hrule \eTD
  \bTD \hrule \eTD
  \bTD 10\high{0} \eTD
  \bTD 1 \eTD
\eTR
\bTR
  \bTD deci \eTD
  \bTD d \eTD
  \bTD 10\high{-1} \eTD
  \bTD $\displaystyle{\fraction{1}{10}}$ \eTD
\eTR
\bTR
  \bTD centi \eTD
  \bTD c \eTD
  \bTD 10\high{-2} \eTD
  \bTD $\displaystyle{\fraction{1}{100}}$ \eTD
\eTR
\bTR
  \bTD milli \eTD
  \bTD m \eTD
  \bTD 10\high{-3} \eTD
  \bTD $\displaystyle{\fraction{1}{1000}}$ \eTD
\eTR
\bTR
  \bTD mikro \eTD
  \bTD \mu \eTD
  \bTD 10\high{-6} \eTD
  \bTD $\displaystyle{\fraction{1}{1000000}}$ \eTD
\eTR
\bTR
  \bTD nano \eTD
  \bTD n \eTD
  \bTD 10\high{-9} \eTD
  \bTD $\displaystyle{\fraction{1}{1000000000}}$ \eTD
\eTR
\eTABLE
}


\stopbodymatter
\stopproduct
\stopproject
